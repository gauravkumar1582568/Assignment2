\documentclass[journal,12pt,twocolumn]{IEEEtran}
\usepackage{tikz}
\usepackage{amsmath}
\usepackage{breqn}
\usepackage{amssymb}
\pagestyle{empty}
\usepackage{setspace}
\usepackage{gensymb}
\singlespacing
\usepackage{graphicx}
\providecommand{\sbrak}[1]{\ensuremath{{}\left[#1\right]}}
\providecommand{\lsbrak}[1]{\ensuremath{{}\left[#1\right.}}
\providecommand{\rsbrak}[1]{\ensuremath{{}\left.#1\right]}}
\providecommand{\brak}[1]{\ensuremath{\left(#1\right)}}
\providecommand{\lbrak}[1]{\ensuremath{\left(#1\right.}}
\providecommand{\rbrak}[1]{\ensuremath{\left.#1\right)}}
\providecommand{\cbrak}[1]{\ensuremath{\left\{#1\right\}}}
\providecommand{\lcbrak}[1]{\ensuremath{\left\{#1\right.}}
\providecommand{\rcbrak}[1]{\ensuremath{\left.#1\right\}}}
\let\StandardTheFigure\thefigure
\let\vec\mathbf

\usepackage{amsmath}
\usepackage{amsthm}
\begin{document}

\title{
Assignment - 2
}
\author{ Gaurav Kumar Gautam \\SM21MTECH12013}
\maketitle
\newpage
\bigskip
\bibliographystyle{IEEEtran}
\section*{\textbf{Chapter III, Miscellaneous Examples VI}}
\noindent
\textbf{\textsl{Question.13:-Two sides of a parallelogram are formed by the straight lines (i)3x-4y=4, (ii)y=mx and the other two sides are formed by two right lines through the point (5,-1) which are parallel to the lines (i) and (ii). Find the two values of m for which the area of parallelogram is 12}}\\[6pt]

\section*{\textbf{Solution}}

First of all equation are given:- \\

$3\vec{x}-4\vec{y}=4$          ...(i)\\
$\vec{y}=m\vec{x}$             ...(ii)
\\

Slope of Equation (i) and (ii) can be given by $m_1$ and $m_2$ respectively,
$$\textbf{m_1}=\frac{3}{4}$$ and $$\textbf{m_2}=m$$\\
Now it's saying two line which is passing through the point (5,-1), which is parallel to the the lines (i) and (ii), so these lines will also have slope of equation (i) and (ii), which is $m_1$ and $m_2$ respectively\\
Now the equation can be given as:-\\[6pt]
$3\vec{x}-4\vec{y}=19$     ...(iii)\\
$m\vec{x}-\vec{y}=5m+1$    ...(iv)\\
So we have total 4 equation.\\
Constant of equation (i) and (ii) are given as $c_1$ and $c_2$,\\
$$\textbf{c_1=-1}$$ and $$\textbf{c_2=0}$$\\
Constant of equation (iii) and (iv) are given as $d_1$ and $d_2$,\\
$$d_1=\frac{-19}{4}$$ and $$d_2=-(5m+1)$$\\
Area of parallelogram can be given as $||\frac{(c_1-d_1)(c_2-d_2)}{(m_1-m_2)}||$, which is equal to 12\\
Now it can be written as:-\\
$$||\frac{(c_1-d_1)(c_2-d_2)}{(m_1-m_2)}||=12$$\\

By putting the values of $c_1$, $c_2$, $d_1$, $d_2$, $m_1$ and $m_2$ in above equation and solving for positive sign of modulus, we get:-\\
$$\frac{(-1-\frac{-19}{4})(o+(5m+1))}{(\frac{3}{4}-m)}=12$$\\
after solving this, value of m we get:\\
$$\textbf{m}=\frac{21}{123}$$\\

Similarly by taking the negative sign of modulus, then after solving for m , we get the value of m:-\\
$$\textbf{m}=\frac{-17}{9}$$\\

So these are the two values of m for which the area of parallelogram is 12.



\begin{figure}
\begin{center}
    \includegraphics[width=.6\textwidth]{assin2.jpeg}
    \caption{Paralellogram}

\end{center}
\end{figure}




\end{document}